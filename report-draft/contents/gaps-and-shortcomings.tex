\section{Gaps \& Shortcomings in Current Research}

\paragraph{} Despite meaningful progress in reverse logistics network design, there are still several key limitations in the current research that deserve further attention.

\begin{itemize}[label=,leftmargin=2mm]
    \item \textbf{Limited Range of Uncertainty Considered::}
        Many studies address uncertainty, but often only in terms of return quantities or timing. Other critical sources of variability—like the quality of returned products, fluctuations in processing outcomes, or disruptions in transportation—are either oversimplified or ignored. For instance, while a few models assume discrete quality grades, the unpredictable nature of return quality is rarely modeled in depth. Similarly, real-world issues like delays in spare parts or vehicle availability are often left out.

    \item \textbf{Static, One-Time Decision Frameworks:}
        Most models are built for single-period planning or assume static conditions. This doesn’t reflect how logistics networks evolve in practice. There’s a lack of dynamic frameworks where decisions can adapt over time in response to shifting demands, costs, or product types.

    \item \textbf{Overlooked Environmental and Social Metrics:}
        Although reverse logistics is often tied to sustainability, explicit environmental metrics like carbon emissions, water use, or energy intensity are missing in most models. The same goes for social factors—aside from rare exceptions, topics such as labor practices, community impact, or informal workforce integration are not incorporated into the optimization frameworks.

    \item \textbf{Simplification of Recovery Processes:}
        Different recovery activities—repair, refurbishment, remanufacturing, disassembly, recycling—often get lumped into a single process in models. This doesn’t reflect the unique cost structures, success rates, or resource requirements of each recovery option. As a result, the models may not capture how real systems function.

    \item \textbf{Challenges in Scalability and Application:}
        Many of the proposed models, especially those involving uncertainty or multi-objective goals, are computationally heavy. This limits their usefulness in large, real-world settings unless more efficient or approximate solution methods are developed.
\end{itemize}
\paragraph{} These gaps highlight that while significant progress has been made, the field of reverse logistics network design still has room for improvement.