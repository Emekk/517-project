\section{Review of Reverse Logistics Network Design Models}

\paragraph{} This section presents a detailed review of 10 selected articles focusing on network design for reverse logistics. The articles are clustered into three clusters based on their approach to uncertainty: Deterministic Models, Stochastic Models, and Robust/Possibilistic Models. Within each cluster, the papers are analyzed in terms of their objectives, the environment, crucial decision variables, modeling techniques, solution approaches, and their contributions to the field.

\subsection{Deterministic Network Design Models}
\paragraph{} Deterministic models provide a fundamental approach to network design, assuming that all parameters are known with certainty. These models mostly focus on optimizing a single objective, cost minimization or profit maximization. The complexity of these deterministic problems, especially for large-scale networks, often requires the use of heuristic solution approaches.

\subsubsection{Jayaraman, Patterson, \& Rolland (2003): The design of reverse distribution networks: Models and solution procedures}

\paragraph{} Main objective is to develop models and efficient solution procedures for designing cost-minimizing reverse distribution networks for product returns.

\paragraph{} A mixed-integer linear programming (MILP) model is proposed. Due to its NP-hard nature, a heuristic solution methodology combining heuristic concentration and heuristic expansion is also developed.

\paragraph{} For the network design, echelons are collection sites, optional sites (intermediate warehouses/return centers), and re/manufacturing sites. The flows involve products moving from collection sites, potentially through intermediate sites, to re/manufacturing sites. Direct shipment from collection to re/manufacturing is allowed. The decisions include which collection and re/manufacturing sites to open, as well as the allocation of product flows through the network. The objective is to minimize total costs, which include variable transportation and processing costs, along with fixed costs. Constraints include demand, facility capacities, site opening.

\paragraph{} For the solution approach, a heuristic procedure involving: random selection of site subsets, heuristic concentration to identify promising sites, and heuristic expansion for greedy improvement. An alternative deterministic heuristic (Procedure CC) is also presented. Subproblems solved using AMPL/CPLEX.

\paragraph{} Heuristic procedures, especially with heuristic expansion, yield high-quality (often near-optimal) solutions much faster than exact methods for large problems. Heuristic expansion significantly improves initial heuristic solutions.

\paragraph{} Jayaraman et al. (2003) establish a comprehensive MILP for a multi-echelon reverse network, which is a common structure in this cluster. Their contribution lies in not only formulating the problem but also in developing effective heuristics. The model focuses purely on the reverse flow, aiming to minimize costs associated with collection, processing, and site opening.

\subsubsection{Min, Ko, \& Ko (2006): A genetic algorithm approach to developing the multi-echelon reverse logistics network for product returns}

\paragraph{} Main objective is to develop a NLMIP model and a Genetic Algorithm to design a cost-minimizing multi-echelon reverse logistics network for product returns, focusing on the number and location of initial collection sites and centralized return centers (CRCs).

\paragraph{} An NLMIP model is formulated to minimize costs associated with product returns. A Genetic Algorithm is developed as the solution methodology due to the problem's complexity.

\paragraph{} The network design includes echelons such as customers, Initial Collection Points (ICPs), Centralized Return Centers (CRCs), and manufacturers/ repair facilities. Flows involve products moving from customers to ICPs and from ICPs to CRCs. Decisions include which ICPs and CRCs to open, customer assignment to ICPs, product flow quantities from ICPs to CRCs, and the length of collection periods at ICPs. The objective is to minimize total reverse logistics costs, including renting costs for ICPs, inventory carrying opening costs for CRCs, and shipping costs from ICPs to CRCs. Constraints include customer assignment, flow through open facilities, flow balance at ICPs, CRC capacity, maximum customer to ICP distance, and minimum numbers of open ICPs and CRCs.

\paragraph{} A Genetic Algorithm with specific encoding for facility opening and collection periods, along with cloning, parent selection, crossover, and mutation operators.

\paragraph{} The Genetic Algorithm effectively solves the NLMIP. Sensitivity analyses indicated that longer consolidation periods at ICPs reduce total costs, increased allowable customer-to-ICP distance reduces costs and ICP numbers, and total costs are sensitive to unit inventory carrying costs.

\paragraph{} Similar to Jayaraman et al., this paper considers a deterministic, cost-minimization problem for general returns. However, it introduces nonlinearity in the objective function to capture the relationship between inventory holding costs and order volume (enabling economies of scale).

\subsubsection{Srivastava (2008): Network design for reverse logistics}

\paragraph{} Main objective is to develop a mathematical model for designing a cost-effective and efficient multi-echelon, multi-period reverse logistics and value recovery network for product returns, specifically within the Indian context, emphasizing Green Supply Chain Management (GrSCM) principles.

\paragraph{} The study combines descriptive modeling with optimization techniques. A hierarchical optimization approach is used. The first level optimizes collection center locations, and the second one, main one, then determines disposition decisions, rework facility locations, capacities, and product routing. The model is formulated as an MILP.

\paragraph{} The network design includes echelons such as customers, collection centers, re/manufacturing, and markets (primary/secondary). Flows involve multi-product returns of various grades, moving from customers to collection centers, then to re/manufacturing sites or sold directly, with reworked products/modules sent to markets. Decisions include disposition at collection centers, location of re/manufacturing sites, capacity additions at re/manufacturing sites, and flows between echelons. The objective is to maximize profit (reselling revenue minus reverse logistics costs and resolution price). Constraints include disposition logic, capacity balance, processing within limits, and inventory balance.

\paragraph{} Hierarchical: Simple optimization for collection center location. Main MILP for disposition, re/manufacturing facility configuration, and flows.

\paragraph{} Remanufacturing often not economically viable in the Indian context; refurbishing dominates. Customer convenience and transport costs significantly impact network.

\paragraph{} This makes a hierarchical approach and a broad application in a developing economy. While the core optimization models are deterministic MILPs, the overall framework considers a wider range of product types. The separation of decisions into levels is different from the other papers in this cluster.

\subsubsection{Qian, Han, Da, \& Stokes (2012): Reverse logistics network design model based on e-commerce}

\paragraph{} Main objective is to study and propose a reverse logistics network design model specifically for e-commerce, aiming to minimize overall logistic costs by determining optimal locations for factories, online retailers, and third-party logistics providers (3PLs).

\paragraph{} A binary MILP model is proposed. Demand/return determination is a separate step. A case study illustrates the model.

\paragraph{} The network design includes echelons such as factories, online retailers, 3PLs, and consuming markets. Flows involve forward flows from factories/3PLs to online retailers to markets, and reverse flows from markets to 3PLs to factories/online retailers. Decisions include which factories, online retailers, and 3PLs to open/use, flow fractions for demand and returns. The objective is to minimize total logistical costs. Constraints include flow balance, facility opening logic, capacity limits, and minimum return portion to factories.

\paragraph{} Model identifies optimal locations/flows. 3PLs centralize returns, routing to online retailers or factories.

\paragraph{} Qian et al. (2012) bring a specific application context (e-commerce) to the deterministic modeling cluster. A distinctive feature is the explicit inclusion of 3PLs as central collectors, reflecting a common strategy in e-commerce reverse logistics. While demand and return determination is discussed as a separate, potentially uncertainty-laden step, the network design model itself uses these as deterministic inputs.

\subsubsection{Ferri, Chaves, \& Ribeiro (2015): Reverse logistics network for municipal solid waste management: The inclusion of waste pickers as a Brazilian legal requirement}

\paragraph{} Main objective is to propose and validate a reverse logistics network model for Municipal Solid Waste (MSW) management in Brazil, maximizing profit while incorporating legal requirements for waste picker inclusion and strategic Material Recovery Facility (MRF) allocation. An MILP model is developed, validated via scenario analysis (São Mateus, Brazil).

\paragraph{} For the network design, echelons include MSW generation location MRFs, waste picker associations, and final destinations. Flows involve recyclable (MSW$_{R}$) and general (MSW$_{G}$) waste to MRFs, sorted materials to recyclers/dealers, and refuse to landfills. Decisions include the number, location, and capacity of MRFs, as well as flow allocations. The objective is to maximize total profit. Constraints ensure all waste is processed, respect MRF capacities, maintain flow conservation, and adhere to facility opening logic.

\paragraph{} Model optimizes MRF configuration. MRF inclusion (especially with waste pickers) reduces transport costs and helps formalization. Higher selective collection improves profitability.

\paragraph{} This paper is notable for integrating social and legal constraints into a deterministic MILP framework for MSW. This work highlights how deterministic models can be adapted to specific socio-legal contexts.

\paragraph{} To summarize the deterministic models, they provided essential building blocks for reverse logistics network design. They typically employ MILP or MINLP formulations to optimize objectives like cost minimization or profit maximization. Key decisions revolve around facility location and flows. Due to the combinatorial complexity, especially with many potential locations, heuristic and metaheuristic solution approaches are common. While these models offer clarity and tractability for certain problem scopes, their primary limitation is the assumption of certainty, which do not hold in reality. Contexts range from general product returns to specific applications like e-commerce and MSW management.

\subsection{Stochastic Network Design Models}

\paragraph{} Stochastic models explicitly address uncertainty by incorporating probabilistic information about parameters like demand, return rates, or lead times. These models often aim to optimize the expected performance of the network or to satisfy service levels with a certain probability, providing more robust insights than deterministic approaches.

\subsubsection{Salema, Barbosa-Povoa, \& Novais (2007): An optimization model for the design of a capacitated multi-product reverse logistics network with uncertainty}
\paragraph{} Main objective is to develop a generalized MILP model for designing a capacitated, multi-product reverse logistics network considering uncertainty in product demands and returns, extending Fleischmann et al.'s (2001) RNM. An MILP formulation with a scenario-based approach for uncertainty. The objective minimizes expected total cost.

\paragraph{} For the network design, echelons include factories, warehouses, customers, and disassembly centers. Flows are multi-product, with forward flows from factories to warehouses to customers and reverse flows from customers to disassembly centers to factories or disposal. Decisions include facility locations (factories, warehouses, disassembly centers), flow allocations, and non-satisfied demand/return fractions. The objective is to minimize total expected costs, including fixed opening costs, variable handling costs, and penalties for non-satisfaction. Uncertainty is modeled as scenario-dependent customer demand and return volumes with associated probabilities. Constraints ensure demand/return satisfaction, factory balance (returns $\le$ demand), and facility capacities.

\paragraph{} Model determines optimal network under uncertainty. Capacity constraints and multi-product nature that affect network structure.

\paragraph{} This work is a significant step from deterministic models by using a scenario-based stochastic programming approach to handle demand and return uncertainty in an integrated forward-reverse MILP. The objective is to minimize expected total costs.

\subsubsection{Roghanian \& Pazhoheshfar (2014): An optimization model for reverse logistics network under stochastic environment by using genetic algorithm}

\paragraph{} Main objective is to propose a probabilistic MILP for designing a multi-product, multi-stage reverse logistics network under uncertainty (stochastic demands), minimizing total fixed opening and shipping costs.

\paragraph{} Probabilistic MILP where demands are random variables. Converted to a deterministic equivalent using chance constraints. Solved with a priority-based Genetic Algorithm.

\paragraph{} For the network design, echelons include return centers, disassembly centers, processing centers, manufacturing centers, and recycling centers. Flows are multi-product and multi-part, with products moving from returning to disassembly/processing centers, parts from disassembly to processing/recycling, and processed parts to manufacturing/recycling. Decisions include which disassembly and processing centers to open and the transportation strategy. The objective is to minimize total shipping costs plus fixed opening costs. Stochasticity arises from demands at manufacturing/recycling centers, handled via chance constraints. Constraints include facility capacities, flow balance, probabilistic demand satisfaction, and limits on open facilities.

\paragraph{} Roghanian \& Pazhoheshfar (2014) also consider demand uncertainty, but they use chance-constrained programming, converting the probabilistic model into a deterministic equivalent. This ensures that demand satisfaction constraints are met with a predefined confidence level. A priority-based Genetic Algorithm is used for solution, indicating the computational challenge of such stochastic models.

\subsubsection{Lieckens \& Vandaele (2007): Reverse logistics network design with stochastic lead times}

\paragraph{} Main objective is to design an efficient single-product, single-level reverse logistic network by extending MILP models with queueing theory (G/G/1) to account for stochastic lead times, inventory, and uncertainty, maximizing yearly profit.

\paragraph{} Extends MILP with (G/G/1) queueing approximations for waiting times/inventory, resulting in an MINLP. Solved with a Genetic Algorithm based on Differential Evolution.

\paragraph{} For the network design, echelons include disposer markets (sources), reuse markets (sinks), and re/manufacturing facilities. Flows are single product, moving from disposer markets to recovery facilities and then to reuse markets. Decisions include which recovery facilities to open, their capacity levels, flow allocations, and fractions of unsatisfied demand/uncollected returns. The objective is to maximize total expected yearly profit, accounting for revenue, fixed costs, reprocessing costs, transport costs, disposal penalties, and inventory holding costs. Stochasticity arises from variability in returns and reprocessing times, modeled via the squared coefficient of variation. Queueing approximations (G/G/1) are used to estimate expected waiting times and inventory levels, leading to nonlinear inventory costs. Little's Law links cycle time to inventory. Constraints include flow balance, demand satisfaction, facility capacities, and logical opening constraints.

\paragraph{} Incorporating queueing (inventory costs from lead times) yields different network configurations than traditional MILPs. Model relevant when inventory/lead time costs are significant. Differential Evolution finds robust, near-optimal solutions.

\paragraph{} This paper introduces a different type of stochasticity and operational variability (lead times, processing times) rather than just quantity uncertainty. The objective is profit maximization, and a Differential Evolution based GA is used to solve the resulting MINLP. This highlights the need for specialized models when operational dynamics are critical.

\paragraph{} The stochastic models in this cluster represent a significant advancement by explicitly incorporating uncertainty, primarily in demand/return quantities or operational times. They typically optimize an expected value or ensure probabilistic constraint satisfaction. These models often result in more complex formulations and require complex solution techniques like metaheuristics or specialized algorithms for scenario-based problems. They provide more realistic and potentially more robust network designs compared to deterministic counterparts.

\subsection{Robust and Possibilistic Network Design Models}
\paragraph{} When precise probability distributions for uncertain parameters are difficult to obtain, robust optimization and possibilistic programming offer alternative paradigms. Robust optimization seeks solutions that are immune to uncertainty or perform well under worst-case conditions, often without requiring probability data.

\subsubsection{Ramezani, Bashiri, \& Tavakkoli-Moghaddam (2013): A robust design for a closed-loop supply chain network under an uncertain environment}

\paragraph{} Main objective is to present a robust design for a multi-product, multi-echelon closed-loop logistic network where demand and return rates are uncertain (finite set of scenarios), maximizing total profit using a min-max regret criterion.

\paragraph{} The network design is formulated such that the echelons include suppliers, plants, distribution centers, customers, collection centers, repair centers, and disposal centers. Flows are multi-product, with forward flows from suppliers to plants to distribution centers to customers and reverse flows from customers to collection centers to repair/disposal centers. Decisions include facility location and capacity levels (plants, distribution, collection, repair centers). The objective is to minimize the maximum regret across scenarios. Uncertainty is modeled via finite scenarios for demand and return rates. Constraints include flow balance, facility capacities, maximum number of open facilities, binary/non-negativity.

\paragraph{} Robust configuration is more reliable (feasible across more scenarios) than deterministic, though potentially less profitable under ideal conditions. Scenario relaxation is computationally superior to extensive form.

\paragraph{} This paper exemplifies the robust optimization approach by aiming to minimize the maximum regret, ensuring that the chosen network design performs well relative to the optimal solution under any realized scenario of demand and return rates. A scenario relaxation algorithm is used to make the problem computationally solvable.

\subsubsection{Hamidieh \& Fazli-Khalaf (2017): A Possibilistic Reliable and Responsive Closed Loop Supply Chain Network Design Model under Uncertainty}

\paragraph{} Main objective is to design a bi-objective, reliable, and responsive multi-echelon closed-loop supply chain network minimizing total network costs and total earliness/tardiness, considering uncertainty (possibilistic programming) and facility disruptions (scenario-based).

\paragraph{} A multi-objective possibilistic programming for uncertain parameters, converted to crisp equivalent. Scenario-based approach for distribution center disruptions. Solved with epsilon-constraint method.

\paragraph{} For the network design, echelons include suppliers, plants, distribution centers, customer zones, collection/inspection centers, refurbishing centers, recycling centers, and disposal centers. Flows are single product, with forward flows from suppliers to plants to distribution centers to customer zones and reverse flows from customer zones to collection/inspection centers to refurbishing/recycling/disposal centers. Decisions include facility location and capacity levels, flow quantities, and earliness/tardiness. The objectives are to minimize total network costs and total weighted earliness/tardiness (scenario-weighted). Uncertainty is modeled as triangular fuzzy numbers for costs/demands. Constraints include demand satisfaction, flow balance, facility capacities, quality, binary opening, non-negativity.

\paragraph{} This work uses possibilistic programming, which is useful when uncertainty is based on subjective expert knowledge rather than historical data. It models uncertain parameters as triangular fuzzy numbers. A key feature is its bi-objective nature, minimizing both total costs and earliness/tardiness, thus explicitly considering network responsiveness alongside economic efficiency and reliability (through disruption scenarios).


\paragraph{} The articles in this cluster address uncertainty when probabilistic information is limited or unreliable. Robust optimization focuses on performance guarantees against worst-case scenarios, often leading to more conservative but reliable designs. Possibilistic programming offers a way to incorporate expert-based fuzzy uncertainty. These models are computationally demanding and often require specialized algorithms like scenario relaxation or multi-objective solution techniques. They are particularly valuable for strategic, long-term network design where downside risk protection is paramount.
