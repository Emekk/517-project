\section{Introduction}

Reverse Logistics (RL) has rapidly evolved from a niche concern to a strategic imperative for modern enterprises. Driven by a confluence of factors including heightened environmental awareness, stringent legislative mandates (such as Extended Producer Responsibility), the economic potential of value recovery from used products, and the pursuit of circular economy principles, RL is reshaping supply chain management. The effective and efficient handling of product returns—encompassing a diverse range of activities from initial collection and inspection to various recovery options like repair, refurbishment, remanufacturing, component harvesting, and material recycling, culminating in responsible disposal—is critically dependent on the sophisticated design of the underlying logistics network.

Designing these reverse networks, however, presents a distinct and often more complex set of challenges compared to traditional forward logistics. A primary complicating factor is the inherent and pervasive uncertainty associated with the entire returns process. This includes variability in the timing of returns, the quantity of products returned, the quality and condition of these items (which dictates feasible recovery paths and their associated costs and yields), and the market dynamics for recovered materials or components. Furthermore, the operational scope can be vast, involving multiple product types, diverse stakeholders, and multi-echelon network structures that must be optimized.

This report undertakes a focused review of contemporary academic literature pertaining to the design of reverse logistics networks. The central aim is to map the current research landscape, identify predominant methodological approaches, critically assess the existing gaps and shortcomings within these approaches, and, consequently, to propose promising and pertinent directions for future research in this dynamic field.

To provide a clear and structured analysis, the ten selected research articles forming the basis of this review are primarily clustered according to their fundamental approach to modeling and managing uncertainty. This dimension is crucial as it significantly influences model complexity, solution methodology, and the practical applicability of the derived network designs. The main clusters employed for this review are:

\begin{enumerate}
    \item \textbf{Deterministic Network Design Models:} This category includes models that operate under the foundational assumption that all relevant parameters influencing the network design (such as return volumes, processing costs, and facility capacities) are known with certainty or are treated as fixed, point estimates for the purpose of optimization.
    \item \textbf{Stochastic Network Design Models:} This group encompasses models that explicitly acknowledge and incorporate uncertainty by utilizing probabilistic methods. This often involves representing key parameters as random variables with known probability distributions or employing scenario-based planning where different potential future states are assigned specific probabilities of occurrence.
    \item \textbf{Robust and Possibilistic Network Design Models:} This cluster comprises models designed to yield network configurations that are resilient and perform adequately across a range of possible (and often imprecisely defined) uncertain conditions. Robust optimization techniques often focus on worst-case performance or minimizing regret, while possibilistic programming leverages fuzzy set theory to handle parameters described by possibility distributions, frequently derived from expert elicitation.
\end{enumerate}

By examining the literature through this lens of uncertainty management, this report seeks to provide valuable insights for both academics and practitioners striving to develop more effective, resilient, and sustainable reverse logistics networks. The subsequent sections will delve into the detailed review of articles within these clusters, followed by a synthesis of gaps, an outline of future research avenues, and concluding remarks.