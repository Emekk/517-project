\section{Introduction}

\paragraph{} Closed Loop Supply Chain (CLSC) has become more important for companies in recent years. It is no longer just a small part of the supply chain but a crucial part of business strategy. This change is led by many reasons, such as more awareness about the environment, stricter laws like Extended Producer Responsibility (EPR), the chance to earn money from used products, and the idea of a circular economy. CLSC helps firms handle returned products in a better way. This includes steps like collecting, checking, repairing, refurbishing, remanufacturing, reusing parts, recycling, or safely disposing of items. To manage all these steps well, firms need a powerful and well-planned reverse logistics network.

\paragraph{} Although forward logistics can be complicated, designing reverse logistics networks is even more difficult. This is mainly because it is hard to know exactly when items will be returned, how many will come back, and what their condition will be. These things affect how the returned items can be used, how much the process will cost, and what value can be gained from them. Also, reverse logistics often deals with many various item types, people, and levels in the supply chain, which makes the system even more complex.
This report looks at 10 academic papers about how to design reverse logistics networks. The main goal is to understand what has been studied so far, what methods are often used, where the gaps are, and what could be researched in the future. To make the review easier to follow, ten selected articles are grouped based on how they deal with uncertainty. This is significant because it affects how the models work, how hard they are to solve, and how helpful they are in real life.
The models under consideration are clustered into three categories:
\begin{itemize}[label=, leftmargin=2mm]
    \item \textbf{Deterministic Models:} These operate on the hypothesis of complete certainty regarding crucial variables such as return quantities, processing costs, and facility capacities.

    \item \textbf{Stochastic Models:} These address uncertainty by incorporating probabilities or scenario based analyses to predict possible future researches.
    \item \textbf{Robust Models and Probabilistic Models:} This group focuses on maintaining system performance despite uncertainty. Robust models are designed to minimize risk or ready for worst case scenarios. In contrast, possibilistic models utilize fuzzy logic to handle imprecise or ambiguous data, often incorporating expert opinion.
\end{itemize}

\paragraph{} Using this categorization to analyze the existing literature, this report seeks to support researchers and practitioners in creating resilient, efficient, and safe for the environment reverse logistics systems. The following sections will check appropriate articles for each model type, highlight current research gaps, propose options for future researchs, and conclude by synthesizing crucial findings.
