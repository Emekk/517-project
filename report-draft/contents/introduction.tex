\section{Introduction}

\paragraph{} Closed Loop Supply Chain (CLSC) is a crucial part of business strategy thanks to awareness about the environment, stricter laws like Extended Producer Responsibility (EPR), the chance to earn money from used products, and the idea of a circular economy. CLSC helps firms handle returned products in a better way. This includes steps like collecting, repairing, refurbishing, remanufacturing, reusing parts, recycling, or disposing of items. To manage all these steps well, firms need a powerful and well-planned reverse logistics network.

\paragraph{} Although forward logistics can be complicated, designing reverse logistics networks is even more difficult. This is mainly because it is hard to know exactly when items will be returned, how many will come back, and what their condition will be. These things affect how the returned items can be used, how much the process will cost, and what value can be gained from them. Also, reverse logistics often deals with many various item types, people, and levels in the supply chain, which makes the system even more complex.
This report looks at 10 academic papers about how to design reverse logistics networks. The main goal is to understand what has been studied so far, what methods are often used, where the gaps are, and what could be researched in the future. To make the review easier to follow, ten selected articles are grouped based on how they deal with uncertainty. This is significant because it affects how the models work, how hard they are to solve, and how helpful they are in real life.
The models reviewed are clustered into three categories:
\begin{itemize}[label=, leftmargin=2mm]
    \item \textbf{Deterministic Models:} assume all information is known in advance,
    \item \textbf{Stochastic Models:} handle randomness through probabilities or scenarios,
    \item \textbf{Robust Models and Probabilistic Models:} prepare for worst-case conditions or rely on fuzzy logic when precise data isn't available.
\end{itemize}

\paragraph{} Using this categorization to analyze the existing literature, this report points out ideas on creating resilient and efficient reverse logistics systems. The following sections will check the articles for each model type, highlight current research gaps, and propose options for future researchs.
